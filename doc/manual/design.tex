\chapter{Design Decisions}
During the implementation of {\ViennaCL}, several design decisions have been
necessary, which are often a trade-off among various advantages and
disadvantages. In the following, we discuss several design decisions and their alternatives.

\section{Transfer CPU-GPU-CPU for Scalars}
The {\ViennaCL} scalar type \lstinline|scalar<>| essentially behaves like a CPU
scalar in order to make any access to GPU ressources as simple as possible, for example
\begin{lstlisting}
  float cpu_float = 1.0f;
  viennacl::linalg::scalar<float> gpu_float = cpu_float;

  gpu_float = gpu_float * gpu_float;
  gpu_float -= cpu_float;
  cpu_float = gpu_float;
\end{lstlisting}
As an alternative, the user could have been required to use \lstinline|copy| as
for the vector and matrix classes, but this would unnecessarily complicate many
commonly used operations like
\begin{lstlisting}
  if (norm_2(gpu_vector) < 1e-10) { ... }
\end{lstlisting}
or
\begin{lstlisting}
  gpu_vector[0] = 2.0f;
\end{lstlisting}
where one of the operands resides on the CPU and the other on the GPU.
Initialization of a separate type followed by a call to \lstinline|copy| is
certainly not desired for the above examples.

However, one should use \lstinline|scalar<>| with care, because the
overhead for transfers from CPU to GPU and vice versa is very large for the
simple \lstinline|scalar<>| type.

\NOTE{Use \lstinline|scalar<>| with care, it is much slower than built-in types
on the CPU!}


\section{Transfer CPU-GPU-CPU for Vectors}

The present way of data transfer for vectors and matrices from CPU to GPU to CPU
is to use the provided \lstinline|copy| function, which is similar to its
counterpart in the Standard Template Library (STL):
\begin{lstlisting}
 std::vector<float> cpu_vector(10);
 ViennaCL::LinAlg::vector<float> gpu_vector(10);

 /* fill cpu_vector here */

 //transfer values to gpu:
 copy(cpu_vector.begin(), cpu_vector.end(), gpu_vector.begin());

 /* compute something on GPU here */

 //transfer back to cpu:
 copy(gpu_vector.begin(), gpu_vector.end(), cpu_vector.begin());
\end{lstlisting}
A first alternative
approach would have been to to overload the assignment operator like this:
\begin{lstlisting}
 //transfer values to gpu:
 gpu_vector = cpu_vector;

 /* compute something on GPU here */

 //transfer back to cpu:
 cpu_vector = gpu_vector;
\end{lstlisting}
The first overload can be directly applied to the \lstinline|vector|-class
provided by \ViennaCL. However, the question of accessing data in the
\lstinline|cpu_vector| object arises. For \lstinline|std::vector| and C arrays, the bracket
operator can be used, but the parenthesis operator cannot. However, other vector types
may not provide a bracket operator. Using STL iterators is thus the more reliable variant.

The transfer from GPU to CPU would require to overload the
assignment operator for the CPU class, which cannot be done by {\ViennaCL}. Thus,
the only possibility within {\ViennaCL} is to provide conversion operators.
Since many different libraries could be used in principle, the only possibility
is to provide conversion of the form
\begin{lstlisting}
  template <typename T>
  operator T() {/* implementation here */}
\end{lstlisting}
for the types in {\ViennaCL}. However, this would allow even totally meaningless
conversions, e.g.~from a GPU vector to a CPU boolean and may result in obscure unexpected behavior.

Moreover, with the use of \texttt{copy} functions it is much clearer, at which
point in the source code large amounts of data are transferred between CPU and GPU.

\section{Solver Interface}
We decided to provide an interface compatible to {\ublas} for dense matrix
operations. The only possible generalization for iterative solvers was to use
the tagging facility for the specification of the desired iterative solver.

\section{Iterators}
Since we use the iterator-driven \lstinline|copy| function for transfer from
CPU to GPU to CPU, iterators have to be provided anyway. However, it has to be repeated
that they are usually VERY slow, because each data access (i.e.~dereferentiation) implies a new
transfer between CPU and GPU. Nevertheless, CPU-cached vector and matrix
classes could be introduced in future releases of {\ViennaCL}.

A remedy for quick iteration over the entries of e.g.~a vector is the following:
\begin{lstlisting}
 std::vector<double> temp(gpu_vector.size());
 copy(gpu_vector.begin(), gpu_vector.end(), temp.begin());
 for (std::vector<double>::iterator it = temp.begin();
      it != temp.end();
     ++it)
 {
  //do something with the data here
 }
 copy(temp.begin(), temp.end(), gpu_vector.begin());
\end{lstlisting}
The three extra code lines can be wrapped into a separate iterator class by the library user, who also has to ensure data consistency during the loop.


\section{Initialization of Compute Kernels}
Since {\OpenCL} relies on passing the {\OpenCL} source code to a
built-in just-in-time compiler at run time, the necessary kernels have to be generated every
time an application using {\ViennaCL} is started.

One possibility was to require a mandatory
\begin{lstlisting}
 viennacl::init();
\end{lstlisting}
before using any other objects provided by {\ViennaCL}, but this approach was discarded for the following two reasons:
\begin{itemize}
 \item If \lstinline|viennacl::init();| is accidentally forgotten by the user,
the program will most likely terminate in a rather uncontrolled way.
 \item It requires the user to remember and write one extra line of code, even
if the default settings are fine.
\end{itemize}
Initialization is instead done in a lazy manner when requesting {\OpenCL} kernels.
Kernels with similar functionality are grouped together in a common compilation units.
This allows a fine-grained control over which source code to compile
where and when. For example, there is no reason to compile the sparse matrix
compute kernels at program startup if there are no sparse matrices used at all.

Moreover, the just-in-time compilation of all available compute kernels in
{\ViennaCL} takes several seconds. Therefore, a request-based compilation is used to minimize any overhead due to just-in-time compilation.

The request-based compilation is a two-step process: At the first instantiation
of an object of a particular type from {\ViennaCL}, the full source code for all objects of the same type is
compiled into a {\OpenCL} program for that type. Each program contains plenty of compute kernels, which are not yet
initialized. Only if an argument for a compute kernel is set, the kernel
actually cares about its own initialization. Any subsequent calls of that
kernel reuse the already compiled and initialized compute kernel.

\NOTE{When benchmarking {\ViennaCL}, first a dummy call to the functionality of interest should be issued prior to taking timings. Otherwise, benchmark results include the just-in-time compilation, which is a constant independent of the data size.}

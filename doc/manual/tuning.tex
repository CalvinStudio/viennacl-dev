\chapter{OpenCL Kernel Parameter Tuning} \label{chap:tuning}
The choice of the global and local work sizes for {\OpenCL} kernels typically has a considerable impact on the obtained device performance.
The default setting in {\ViennaCL} is -- with some exceptions -- to use the same number of work groups and work items per work group (128) for each compute kernel.
To obtain highest performance, optimal work sizes have to be determined for each kernel in dependence of the underlying device.

\section{Start Tuning Runs}
{\ViennaCLversion} ships with a automated tuning environment, which tries to determine the best kernel parameters for the available device.
At present, only kernel parameters for the first device are optimized. The tuning programs are located in
\begin{itemize}
 \item \texttt{examples/parameters/vector.cpp}: Tuning for vector kernels
 \item \texttt{examples/parameters/matrix.cpp}: Tuning for matrix kernels
 \item \texttt{examples/parameters/sparse.cpp}: Tuning for sparse matrix kernels
\end{itemize}
and are built together with other examples when using {\CMake}. The executables are
\begin{itemize}
 \item \texttt{vectorparams},
 \item \texttt{matrixparams},
 \item \texttt{sparseparams}
\end{itemize}
respectively and are executed without additional parameters.
During execution, these programs create three XML files \texttt{vector\_parameters.xml}, \texttt{matrix\_parameters.xml} and \texttt{sparse\_parameters.xml}, which hold the best parameter set.

At present, only {\ViennaCL} types with standard alignment are benchmarked. Higher performance can be obtained when allowing further memory alignments and comparing different implementations.
This, however, is not yet available, but may be part of future versions.

\section{Load Best Parameters at Startup}
In order to load the best parameters at each startup, the parameter reader located at \texttt{viennacl/io/kernel\_parameters.hpp} can be used.
The individual kernels for the respective {\ViennaCL} types can be loaded with the lines
\begin{lstlisting}
using viennacl::io;
read_kernel_parameters< viennacl::vector<float> >("vector_parameters.xml");
read_kernel_parameters< viennacl::matrix<float> >("matrix_parameters.xml");
read_kernel_parameters< viennacl::compressed_matrix<float> >("sparse_parameters.xml");

//similarly for the numeric type double
\end{lstlisting}
where the filename is as usual relative to current working directory. A simple example doing just that can be found in \texttt{examples/parameters/parameter\_reader.cpp}.
In principle, kernel parameters can all be located in a single XML file, from which the call to \lstinline|read_kernel_parameters()| will then extract the relevant ones for the respective
{\ViennaCL} type and the available device.

\TIP{Please note that in order to read the parameters, the project has to be linked with \texttt{pugixml} \cite{pugixml}, which is shipped with {\ViennaCL} in \texttt{external/} }

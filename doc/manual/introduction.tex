
\chapter*{Introduction}   \addcontentsline{toc}{chapter}{Introduction}

The Vienna Computing Library (\ViennaCL) is a scientific computing
library written in C++. It allows simple, high-level access
to the vast computing resources available on parallel architectures such as
GPUs and multi-core CPUs by using either a host-based computing backend, an {\OpenCL} computing backend, or {\CUDA}.
The primary focus is on common linear algebra
operations (BLAS levels 1, 2 and 3) and the solution of large sparse systems of equations by means of iterative
methods. In {\ViennaCLminorversion}, the following iterative solvers are
implemented (confer for example to the book of Y.~Saad \cite{saad-iterative-solution}):
\begin{itemize}
 \item Conjugate Gradient (CG)
 \item Stabilized BiConjugate Gradient (BiCGStab)
 \item Generalized Minimum Residual (GMRES)
\end{itemize}
A number of preconditioners is provided with {\ViennaCLversion} in order to improve convergence of these solvers, cf.~Chap.~\ref{chap:algorithms}.

The solvers and preconditioners can also be used with different
libraries due to their generic implementation. At present, it is possible to
use the solvers and preconditioners directly with types from the {\ublas} library, which is part of 
{\Boost} \cite{boost}. The iterative solvers can directly be used with Eigen \cite{eigen} and MTL 4 \cite{mtl4}.

Under the hood, {\ViennaCL} uses a unified layer to access {\CUDA} \cite{nvidiacuda}, {\OpenCL} \cite{khronoscl}, and/or {\OpenMP} \cite{openmp} for accessing and
executing code on compute devices. Therefore, {\ViennaCL} is not tailored 
to products from a particular vendor and can be used on many different
platforms. At present, {\ViennaCL} is known to work on all current CPUs and modern GPUs from NVIDIA
and AMD (see Tab.~\ref{tab:double-precision-GPUs}), CPUs 
using either the AMD Accelerated Parallel Processing (APP) SDK (formerly ATI Stream SDK) or the Intel OpenCL SDK, and Intels MIC platform (Xeon Phi).

\NOTE{Double precision arithmetic on GPUs is only possible if it is provided by the GPU. There is no double precision emulation in {\ViennaCL}.}

\NOTE{Double precision arithmetic using the ATI Stream SDK or AMD APP SDK may not be fully
OpenCL-certified. Also, we have observed bugs in AMD APP SDKs 2.7 which effects some algorithms in {\ViennaCL} (e.g.~BiCGStab).}

\begin{table}[tb]
\begin{center}
\begin{tabular}{l|c|c}
Compute Device & float & double \\
\hline
\NVIDIA Geforce 86XX GT/GSO   & ok & - \\
\NVIDIA Geforce 88XX GTX/GTS  & ok & - \\
\NVIDIA Geforce 96XX GT/GSO   & ok & - \\
\NVIDIA Geforce 98XX GTX/GTS  & ok & - \\
\NVIDIA GT 230     & ok & - \\
\NVIDIA GT(S) 240  & ok & - \\
\NVIDIA GTS 250    & ok & - \\
\NVIDIA GTX 260    & ok & ok \\
\NVIDIA GTX 275    & ok & ok \\
\NVIDIA GTX 280    & ok & ok \\
\NVIDIA GTX 285    & ok & ok \\
\NVIDIA GTX 4XX    & ok & ok \\
\NVIDIA GTX 5XX    & ok & ok \\
\NVIDIA GTX 6XX    & ok & ok \\
\NVIDIA Quadro FX 46XX & ok & - \\
\NVIDIA Quadro FX 48XX & ok & ok \\
\NVIDIA Quadro FX 56XX & ok & - \\
\NVIDIA Quadro FX 58XX & ok & ok \\
\NVIDIA Tesla 870    & ok & - \\
\NVIDIA Tesla C10XX  & ok & ok \\
\NVIDIA Tesla C20XX  & ok & ok \\
\hline
ATI Radeon HD 4XXX   & ok & - \\
ATI Radeon HD 48XX   & ok & essentially ok \\
ATI Radeon HD 5XXX   & ok & - \\
ATI Radeon HD 58XX   & ok & essentially ok \\
ATI Radeon HD 59XX   & ok & essentially ok \\
ATI Radeon HD 68XX   & ok & - \\
ATI Radeon HD 69XX   & ok & essentially ok \\
ATI Radeon HD 77XX   & ok & - \\
ATI Radeon HD 78XX   & ok & - \\
ATI Radeon HD 79XX   & ok & essentially ok \\
ATI FireStream V92XX & ok & essentially ok \\
ATI FirePro V78XX    & ok & essentially ok \\
ATI FirePro V87XX    & ok & essentially ok \\
ATI FirePro V88XX    & ok & essentially ok \\
\end{tabular}
\caption{Available arithmetics in {\ViennaCL} provided by selected GPUs. At the
release of {\ViennaCLversion}, the Stream SDK (APP SDK) from AMD/ATI may not comply to
the {\OpenCL} standard for double precision extensions, and we have observed
problems with the APP SDK 2.7 on Linux in both single- and double-precision.}
\label{tab:double-precision-GPUs}
\end{center}
\end{table}
